\documentclass[11pt,a4paper]{article}
\usepackage{fullpage}
\usepackage[T1]{fontenc} 
\usepackage[utf8]{inputenc}
\usepackage{amsmath}
\usepackage{amssymb}
\usepackage{float}
\usepackage{tabularx}
\usepackage{multirow}
\usepackage{graphicx}
\usepackage{geometry}
\usepackage[table,dvipsnames]{xcolor}
\usepackage[hidelinks]{hyperref}
\usepackage[polish]{babel}
\usepackage{menukeys}
\usepackage{subcaption}

\setlength{\parindent}{0cm}
\setlength{\parskip}{2mm}
\newcolumntype{Y}{>{\centering\arraybackslash}X}
\DeclareMathOperator{\sgn}{sgn}

\begin{document}

\title{Rozpoznawanie człowieka metodami biometrii \\
\Large{
    Projekt 1. --- Rozpoznawanie tęczówki \\
    Raport
}}
\author{Bartłomiej Dach}
\maketitle

Poniższy dokument stanowi sprawozdanie z~implementacji aplikacji dokonującej rozpoznawania człowieka na~podstawie tęczówki wysegmentowanej ze~zdjęć oczu.
W~dokumencie opisano zastosowane metody segmentacji, przetwarzania i~porównywania obrazów tęczówek oraz~zawarto wyniki działania dla~wybranych przykładowych obrazów wejściowych.

\section{Wstęp}

% TODO

\section{Opis aplikacji}

W~ramach projektu stworzony został program okienkowy umożliwiający gromadzenie i~porównywanie obrazów ludzkich oczu.
Program został napisany w~języku Java w~wersji 8.

Do~stworzenia interfejsu użytkownika została wybrana biblioteka JavaFX \cite{javafx}.
Główną motywacją wyboru tej biblioteki były:
\begin{itemize}
    \item zgodność z~wieloma platformami (Windows, Linux, Mac),
    \item wbudowane w~bibliotekę klasy pozwalające na~edycję bitmap poprzez zmianę kolorów pojedynczych pikseli, co~umożliwia samodzielną realizację filtrów, na~których oparta jest operacja segmentacji.
\end{itemize}

\begin{figure}
    \centering
    \includegraphics[width=\textwidth]{res/img/main-window.PNG}
    \caption{Główne okno zaimplementowanej aplikacji.}
    \label{fig:main-window}
\end{figure}

\begin{figure}
    \centering
    \includegraphics[width=\textwidth]{res/img/recognition-results.PNG}
    \caption{Widok z~wynikami rozpoznawania.
    Możliwe jest porównanie wizualne rozpoznawanego obrazu z~tymi umieszczonymi w~bazie.}
    \label{fig:recognition-results}
\end{figure}

\subsection{Instrukcja obsługi}

% TODO

\section{Opis metody}

% TODO: ogólne informacje

\subsection{Segmentacja obrazu}

% TODO: wstęp że kradzione

Zaimplementowana metoda należy do~grona metod proceduralnych, tj.~stosuje ściśle określoną listę operacji i~nie stosuje metod optymalizacyjnych lub~stochastycznych.
Kolejne kroki przetwarzania wejściowego obrazu w~celu lokalizacji tęczówki ilustruje rysunek~\ref{fig:block-diagram}.
Opis~poszczególnych operacji znajduje~się w~poszczególnych częściach tego~rozdziału.

Ogólnie mówiąc, proces lokalizacji tęczówki opiera~się na~uprzednim zlokalizowaniu źrenicy.
Przy~lokalizacji przyjęto następujące założenia:
\begin{itemize}
    \item Źrenica jest największym ciemnym obszarem na~obrazie wejściowym.
    \item Źrenicę i~tęczówkę można uznać za~okręgi koncentryczne (tj. środek źrenicy jest również środkiem tęczówki).
\end{itemize}
Z~tego względu w~algorytmie stosowane~są dwa łańcuchy operacji, których wyniki używane są do~geometrycznej lokalizacji środka źrenicy oraz promieni: wewnętrznego i~zewnętrznego źrenicy.

\paragraph{Rozciągnięcie histogramu.}
Na~początku algorytmu wykonywane jest rozciągnięcie histogramu poszczególnych kanałów obrazu zgodnie ze~wzorem
$$ I_o = \frac{I_i - I_{\min}}{I_{\max} - I_{\min}} $$
Celem tej~operacji jest maksymalizacja dynamiki obrazów bez~utraty danych.
Po~jej wykonaniu na~każdym kanale jasność wszystkich pikseli znajduje~się dokładnie w~przedziale $[0, 255]$ (zakładając 8-bitową głębię koloru).
Jest to~istotne szczególnie na~etapie progowania, gdzie~błędy przybliżeń progu mogą mieć znaczące znaczenie na~końcowy wynik.

\paragraph{Filtr gaussowski.}
Po~rozciągnięciu histogramu stosowany jest splotowy filtr gaussowski o~rozmiarze elementu~$3 \times 3$, określonego macierzą
$$ F_g = \begin{bmatrix}
    1 & a & 1 \\
    a & a^2 & a \\
    1 & a & 1
\end{bmatrix} $$
gdzie $a > 1$ stanowi parametr filtra.

Filtr gaussowski jest filtrem dolnoprzepustowym, którego zastosowanie powoduje wygładzenie obrazu.
Celem tej operacji w~łańcuchu jest zamaskowanie szczegółów nieistotnych dla~procesu segmentacji, takich, jak m.in.~ziarno czy~pomijalne przebarwienia małych fragmentów obrazu.

Im większa wartość parametru~$a$, tym mniej zauważalny jest efekt rozmycia.
W~przypadku źrenicy, stosowany jest parametr $a = 1.5$, zaś dla~tęczówki --- $a = 1.8$.
Uzasadnieniem tej rozbieżności jest fakt, że na~ogół zewnętrzny brzeg tęczówki jest mniej ostry niż~brzeg źrenicy, zatem aby zapobiec przekłamaniom przy~pomiarze zewnętrznego promienia, stosowane jest mniejsze rozmycie.

\paragraph{Konwersja do~skali szarości.}
Na~podstawie rozmytego obrazu kolorowego stosowane jest przejście do~skali szarości wg~wzoru
$$ I_o = r \cdot I_r + g \cdot I_g + b \cdot I_b $$
gdzie $I_r, I_g, I_b$ reprezentują jasności pikseli na~kanałach odpowiednio: czerwonym, zielonym i~niebieskim, zaś~$r, g, b$ są~współczynnikami konwersji.
Zarówno dla~tęczówki, jak i~dla~źrenicy przyjęto wspólne wartości
$$ r = 0.299, \qquad g = 0.587, \qquad b = 0.114 $$
odwzorowujące luminancję pikseli w~modelu kolorów YCbCr.

\paragraph{Progowanie.}
Po~konwersji do~skali szarości następuje przejście do~obrazu binarnego poprzez wykonanie operacji progowania zgodnie ze~wzorem
$$ I_o = \begin{cases}
    0 & I_i < t \\
    1 & I_i \geq t
\end{cases} $$
gdzie próg $t$ wyznaczany jest na~podstawie poziomów szarości całego obrazu ze~wzoru
$$ t = \frac{1}{d} \sum_{i = 0}^{W - 1} \sum_{j = 0}^{H - 1} I_{ij} $$
w~którym $W, H$ oznaczają wymiary obrazu (odpowiednio szerokość i~wysokość), zaś~$d$ jest nowym parametrem progowania, określającym stosunek przyjętego progu do~średniej jasności obrazu.

Im większa wartość $d$, tym mniejszy przyjęty próg.
Z~tego względu przyjęto wartość $d_p = 5.5$ dla~lokalizacji źrenicy i~$d_i = 1.4$ dla~dopasowania tęczówki.

\paragraph{Operacje morfologiczne.}
Po~progowaniu do~obrazów wynikowych stosowane są~operacje morfologiczne.
Dwoma podstawowymi operacjami morfologicznymi są: dylacja (rozszerzenie) i~erozja.
Obie operacje używają tzw.~elementu strukturalnego, który można przedstawić jako~macierz.
\begin{itemize}
    \item Operacja dylacji obrazu binarnego polega na~rozszerzeniu obiektu (tutaj zakładamy, że obiekt jest czarny na~białym tle).
    Dla~każdego czarnego piksela następuje translacja elementu strukturalnego tak, aby~jego środek znajdował~się w~danym pikselu, po~czym przesunięty element strukturalny dodawany jest do~obrazu wynikowego.
    \item Operacja erozji powoduje zwężenie obiektu.
    Dla~każdego piksela obrazu wyjściowego rozważany jest element strukturalny przesunięty tak, aby~jego środek leżał na~danym pikselu.
    Jeśli w~obrazie wejściowym przesunięty element strukturalny leży na~samych czarnych pikselach, wtedy wyjściowy piksel też ma~kolor czarny; w~przeciwnym wypadku przyjmowany jest kolor biały.
\end{itemize}
W~przypadku zaimplementowanej metody rozważane~są tylko pełne, kwadratowe elementy strukturalne o~ustalonym rozmiarze $k \times k$.

Złożenie operacji dylacji i~erozji pozwala na~zdefiniowanie dwóch innych operacji morfologicznych, w~zależności od~przyjętej kolejności:
\begin{itemize}
    \item Na~operację zamknięcia obrazu składa~się kolejno: dylacja i~erozja.
    Operacja wypełnia luki w~transformowanym obrazie.

    W~opisywanej metodzie zamknięcie stosowane jest pod~detekcję zewnętrznego konturu tęczówki.
    Wybór ten motywowany jest tym, że~tęczówka jest stosunkowo wielobarwna, co~może zakłócić wynik progowania i~wpłynąć negatywnie na~dopasowanie zewnętrznego promienia tęczówki.
    \item Na~operację otwarcia składa~się natomiast kolejno: erozja i~dylacja.
    Operacja usuwa drobny szum i~zakłócenia z~pierwszego planu obrazu.

    Otwarcie stosowane jest do~detekcji źrenicy.
    Do~wyznaczenia lokalizacji źrenicy stosowane są~projekcje --- szum na~obrazie może zakłócić wyznaczone projekcje, tym samym utrudniając zlokalizowanie źrenicy.
    Zastosowanie otwarcia pozwala wyeliminować niepożądane elementy pierwszego planu.
\end{itemize}

\paragraph{Określenie granic źrenicy.}

Po~wykonaniu jednego łańcucha operacji przeznaczonego dla~źrenicy następuje jej lokalizacja.
Odbywa~się ona przy~pomocy projekcji.

Projekcja obrazu binarnego względem jednej z~osi obrazu określa liczbę pikseli czarnych (pierwszego planu) dla~poszczególnych współrzędnych wzdłuż wybranej osi.
Zakładając, że~po~progowaniu i~operacjach morfologicznych na~obrazie zostanie sama~źrenica, to~maksimum na~projekcjach względem obu~osi powinno wskazywać na~środek źrenicy.

Na~niektórych obrazach zdarza~się jednak, że~na~źrenicy znajdują~się refleksy, np.~od~lampy błyskowej urządzenia rejestrującego.
Może to~powodować przesunięcie środka źrenicy, co~źle wpłynie na~detekcję tęczówki z~racji przyjętego założenia o~jej koncentryczności ze~źrenicą.
Z~tego względu punkt wynikający z~projekcji nie~jest uznawany automatycznie za~środek źrenicy, lecz jako~punkt startowy do~dalszych operacji.

Algorytm znajduje granice źrenicy poprzez przeszukiwanie sąsiedztwa punktu startowego.
Przeszukiwanie to~przypomina algorytm wypełniania kubełkowego (ang. \emph{flood-fill}).
Na~tej podstawie możliwe jest wyznaczenie prostokąta ograniczającego (ang. \emph{bounding box}) źrenicy, skąd stosunkowo łatwo można wyznaczyć środek i~promień źrenicy.
Środek źrenicy to~środek prostokąta, natomiast za~promień źrenicy przyjmowana jest średnia z~szerokości i~wysokości prostokąta.

\paragraph{Wyznaczenie promienia tęczówki.}
Zewnętrzny promień tęczówki obliczany jest z~drugiego zbinaryzowanego obrazu z~użyciem transformacji Hougha.
Polega ona na~zliczaniu pikseli z~pierwszego planu, które leżą na~obwodzie okręgów o~środku w~zadanym punkcie, którym w~tym wypadku jest środek źrenicy, i~kolejnych promieniach.
Dla~uproszczenia przyjęto, że~promienie są~zaokrąglane w~dół, tj.
$$ r = \left\lfloor \sqrt{(x - x_c)^2 + (y - y_c)^2} \right\rfloor $$
gdzie $(x, y)$ to~badany punkt pierwszego planu, a~$(x_c, y_c)$ to współrzędne środka.
Wybierany jest promień o~największej liczbie dopasowanych pikseli.

Przykładowy wynik segmentacji wykonanej powyższą metodą znajduje~się na~rysunku~\ref{fig:reference-segmentation}.

\begin{figure}
    \centering
    \includegraphics[height=20cm]{res/img/diagram.pdf}
    \caption{Schemat blokowy opracowanego procesu segmentacji oka.}
    \label{fig:block-diagram}
\end{figure}

\paragraph{Rozwinięcie tęczówki w~prostokąt.}
Program umożliwia również rozwinięcie tęczówki w~prostokątny obraz o~wymiarach $720 \times 400$.
Rozwinięcie to~odbywa~się poprzez przejście ze~współrzędnych polarnych na~kartezjańskie.
Dla~każdego piksela $(\theta,r)$ obrazu wynikowego wyznaczane~są współrzędne tego~punktu na~obrazie wyjściowym ze~wzorów
\begin{align*}
    \varphi &= 2 \pi \frac{\theta}{720} \\
    r' &= \frac{r}{400} (r_i - r_p) + r_p \\
    x &= r' \cos \varphi \\
    y &= r' \sin \varphi
\end{align*}
gdzie $r_i$ to~zewnętrzny promień tęczówki, a~$r_p$ --- wewnętrzny promień tęczówki (promień źrenicy).

Ponieważ wynikowe współrzędne $(x,y)$ są~ułamkowe, finalnie stosowana jest~interpolacja dwuliniowa na~kwadracie ze~wzoru
\begin{align*}
    f(x, y) = & f(\left\lfloor x \right\rfloor, \left\lfloor y \right\rfloor) \cdot (1 - \{ x \}) \cdot (1 - \{ y \}) + \\
    & f(\left\lceil x \right\rceil, \left\lfloor y \right\rfloor) \cdot \{ x \} \cdot (1 - \{ y \}) + \\
    & f(\left\lfloor x \right\rfloor, \left\lceil y \right\rceil) \cdot (1 - \{ x \}) \cdot \{ y \} + \\
    & f(\left\lceil x \right\rceil, \left\lceil y \right\rceil) \cdot \{ x \} \cdot \{ y \}
\end{align*}
gdzie $\{ x \}$ oznacza część ułamkową liczby~$x$: $\{ x \} = x - \left\lceil x \right\rceil$.

\subsection{Rozpoznawanie na~podstawie wysegmentowanej tęczówki}
% TODO

\paragraph{Podział rozwiniętej tęczówki na~pasma.}

\paragraph{Zastosowanie transformaty falkowej Gabora.}

\paragraph{Wyznaczenie kodu tęczówki.}

\paragraph{Porównywanie kodów tęczówek.}

\section{Wyniki działania metody}
% TODO

\section{Podsumowanie}
% TODO

\begin{thebibliography}{9}

    \bibitem{javafx}
        Oracle Corporation,
        ,,OpenJFX'',
        dokumentacja on-line.
        Dostępne: \url{https://wiki.openjdk.java.net/display/OpenJFX/Main}.

    \bibitem{atoz}
        R. Fisher,
        S. Perkins,
        A. Walker,
        E. Wolfart,
        ,,A~to~Z of~Image Processing Concepts''.
        Dostępne: \url{https://homepages.inf.ed.ac.uk/rbf/HIPR2/glossary.htm}.

    \bibitem{mmu}
        MMU Iris Dataset
        [Online].
        Dostępne: \url{http://www.cs.princeton.edu/~andyz/downloads/MMUIrisDatabase.zip}

\end{thebibliography}

\end{document}
